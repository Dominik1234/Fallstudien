\documentclass[a4paper, 11pt]{scrreprt}
\usepackage[utf8]{inputenc}
\usepackage[ngerman]{babel}
\usepackage[T1]{fontenc}
\usepackage{lmodern}
\usepackage{amsmath,amssymb,amstext,amsfonts,mathrsfs}
\usepackage{graphicx}
\usepackage{color}

\usepackage{marginnote}

\pagestyle{headings}

\newtheorem{defi}{Definition}[section]
\newtheorem{prop}[defi]{Proposition}
\newtheorem{satz}[defi]{Satz}
\newtheorem{koro}[defi]{Korollar}
\newtheorem{lemma}[defi]{Lemma}

\newcommand{\RR}{\mathbb{R}}
\newcommand{\EE}{\mathbb{E}}
\newcommand{\NN}{\mathbb{N}}

\newcommand{\student}[1]{\marginnote{{\normalfont\bf #1}}}

\title{Fallstudien der math. Modellbildung}
\author{Manuela Lambacher, Dominik Otto, Andreas Wiedemann}
\date{\today}

\begin{document}
\parindent 0pt

\maketitle
\tableofcontents

\chapter{Wittaker-Shannon-Sampling Theorem}

\section{The Wittaker-Shannon-Sampling Theorem}
\section{Proof of the Theorem}
\section{Meaning, real-life applications and limitations}


\chapter{Das Marchenko-Pastur-Gesetz}

\section{Das Marchenko-Pastur-Gesetz}
\student{Manuela Lambacher, Dominik Otto, Andreas Wiedemann}

Sei \(Y_N\) eine \(N\times M(N)\)-Matrix mit unabhängigen zentrierten Einträgen mit Varianz \(1\),
	\[\sup_{j,k,N} \EE\left[ | Y_N(j,k)|^q\right] = C_q < \infty \qquad \forall q \in \NN\]
und \(M(N) \in \NN\) so, dass
	\[\lim_{N\to\infty} \frac{M(N)}{N} = \alpha \in[1,\infty). \]
Sei weiterhin die Wishart-Matrix gegeben als 
	\[W_N = \frac{1}{N}Y_NY_N^T,\]
und habe die empirische Eigenwertverteilung
	\[L_N = \frac{1}{n} \sum_{j=1}^{N} \delta_{\lambda_j} \]
und das Zustandsdichtemaß \(\overline{L_N} = \EE[L_N]\). Dann gilt die Konvergenz
	\[\overline{L_N} \xrightarrow{\text{w}} f_{\alpha}(x)dx \quad(N\to\infty)\]
im Raum der Wahrscheinlichkeitsmaße auf \(\RR\), wobei
	\[f_{\alpha}(x)=\frac{1}{2\pi x}\sqrt{(x-(1-\sqrt{\alpha})^2_{+}((1+\sqrt{\alpha})^2_{+}} \]

\end{document}