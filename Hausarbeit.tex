\documentclass[a4paper, 11pt]{scrreprt}
\usepackage[utf8]{inputenc}
\usepackage[ngerman]{babel}
\usepackage[T1]{fontenc}
\usepackage{lmodern}
\usepackage{amsmath,amssymb,amstext,amsfonts,mathrsfs}
\usepackage{graphicx}
\usepackage{color}

\usepackage{marginnote}

\pagestyle{headings}

\newtheorem{defi}{Definition}[section]
\newtheorem{prop}[defi]{Proposition}
\newtheorem{satz}[defi]{Satz}
\newtheorem{koro}[defi]{Korollar}
\newtheorem{lemma}[defi]{Lemma}

\newenvironment{beweis}[1][Beweis]{\begin{trivlist}
	\item[\hskip \labelsep {\bfseries #1}]}
	{\end{trivlist}}

\newcommand{\RR}{\mathbb{R}}
\newcommand{\EE}{\mathbb{E}}
\newcommand{\NN}{\mathbb{N}}

\newcommand{\student}[1]{\marginnote{{\normalfont\bf #1}}}

\title{Fallstudien der math. Modellbildung}
\author{Manuela Lambacher, Dominik Otto, Andreas Wiedemann}
\date{\today}

\begin{document}
\parindent 0pt
\maketitle
\tableofcontents

\chapter{Wittaker-Shannon-Sampling Theorem}

\section{The Wittaker-Shannon-Sampling Theorem}
\section{Proof of the Theorem}
\section{Meaning, real-life applications and limitations}

\chapter{Das Marchenko-Pastur-Gesetz}

\section{Das Marchenko-Pastur-Gesetz}

Sei \(Y_N\) eine \(N\times M(N)\)-Matrix mit unabhängigen zentrierten Einträgen mit Varianz \(1\),
	\[\sup_{j,k,N} \EE\left[ | Y_N(j,k)|^q\right] = C_q < \infty \qquad \forall q \in \NN\]
und \(M(N) \in \NN\) so, dass
	\[\lim_{N\to\infty} \frac{M(N)}{N} = \alpha \in[1,\infty). \]
Sei weiterhin die Wishart-Matrix gegeben als 
	\[W_N = \frac{1}{N}Y_NY_N^T,\]
und habe die empirische Eigenwertverteilung
	\[L_N = \frac{1}{n} \sum_{j=1}^{N} \delta_{\lambda_j} \]
und das Zustandsdichtemaß \(\overline{L_N} = \EE[L_N]\). Dann gilt die Konvergenz
	\[\overline{L_N} \xrightarrow{\text{w}} f_{\alpha}(x)dx \quad(N\to\infty)\]
im Raum der Wahrscheinlichkeitsmaße auf \(\RR\), wobei
	\[f_{\alpha}(x)=\frac{1}{2\pi x}\sqrt{(x-(1-\sqrt{\alpha})^2_{+}((1+\sqrt{\alpha})^2_{+}} \]

\newpage
\begin{beweis}
\begin{align*}
		N^{l+1} &\langle \overline{L_N}, x^l \rangle\ 
		= N^{l+1} \cdot \int x^l \overline{L_N}(dx) 
		= N^{l+1} \cdot \frac{1}{N} \cdot \EE[tr(W^l_N)] 
		= N^l \sum_{j_1,...,j_l = 1}^N \EE\left[\prod_{p = 1}^l W_{j_p,j_{p+1}}\right] \\
		&= N^l \sum_{j_1,...,j_l = 1}^N \EE\left[\prod_{p = 1}^l \frac{1}{N} \sum_{k = 1}^{M(N)} Y_N(j_p,k) \cdot Y_N(j_{p+1},k) \right] \\
		&= \sum_{j_1,...,j_l = 1}^N \EE \left[\left(\sum_{k = 1}^{M(N)} Y_N(j_1,k) \cdot Y_N(j_2,k)\right) \cdot \left(\prod_{p = 2}^l \sum_{k = 1}^{M(N)} Y_N(j_p,k) \cdot Y_N(j_{p+1},k) \right) \right] \\
		&= \sum_{j_1,...,j_l = 1}^N \EE\left[	\prod_{p = 2}^l \sum_{k_1,k_2 = 1}^{M(N)} Y_N(j_1,k_1) \cdot Y_N(j_2,k_1) \cdot Y_N(j_p,k_2) \cdot Y_N(j_{p+1},k_2) \right] \\
		&= ... \\
		&= \sum_{j_1,...,j_l = 1}^N \sum_{k_1,...,k_l = 1}^{M(N)} \EE[Y_N(j_1,k_1) Y_N(j_2,k_1) Y_N(j_2,k_2) Y_N(j_3,k_2) ... Y_N(j_l,k_l) Y_N(j_1,k_l)]
\end{align*}



\end{beweis}


\end{document}
